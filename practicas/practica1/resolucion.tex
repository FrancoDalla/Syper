\documentclass{article}
\author{Franco Dalla Gasperina}
\title{Seguridad y privacidad en redes - Practica 1}
\date {Agosto 2025}
\begin{document}
	\maketitle
	\section{¿Cuál es la diferencia entre seguridad y privacidad?}
		Ambos terminos engloban distintos significados. Puede considerarse también que la privacidad es algo englobado dentro de la seguridad incluso.
		La seguridad entonces, podría considerarse como las distintas prácticas para mantener seguro algo(entorno, información, entre otras cosas), mientras que la privacidad consistiría en el estado o la practica para mantenerlas accesibles a los usuarios correspondientes.
		\subsection{Seguridad}
			Sobre la seguridad hay conceptos universales a tener en cuenta, estos son: 
		\begin{itemize}
			\item Mínimo privilegio:
			Este principio consiste en que los permisos otorgados a un sujeto deben estar basados en las necesidades de los mismos.
			
			\item Valores por omisión seguros:
			A menos de que se le asigne un objeto a un individuo explícitamente. Este debe ser denegado en caso de ser solicitado.
			
			\item Economía de mecanismos:
			Los mecanismos de seguridad deben ser lo más simple posible.
			
			\item Mediación completa:
			Todos los accesos a objetos deben poder auditarse. 
		\end{itemize}
		\subsection{Privacidad}
			La privacidad de la información se refiere al derecho
			de los individuos y las organizaciones a controlar cómo
			se recopila, usa, almacena y comparte su
			información personal o confidencial.
	\section{Defina y relacione los conceptos de vulnerabilidad, amenaza e incidente.}
		\begin{itemize}
			\item Confidencialidad: Garantiza que la información sólo sea
			accesible por las personas autorizadas.
			
			\item Integridad: Garantiza que la información sólo pueda ser
			modificada por quien está autorizado a hacerlo
			
			\item Disponibilidad: Garantiza que los usuarios autorizados
			tienen acceso a la información y recursos relacionados
			cuando lo necesiten. 
		\end{itemize}
	\section{¿A qué se hace referencia con el concepto de “seguridad en capas” y “zero trust”? ¿Considera que están relacionados?}
		\subsection{seguridad en capas}
			La seguridad en capas consiste en organizar las practicas de seguridad para cada capa del modelo tradicional de internet. 
			Un enfoque posible seria el siguiente:
			\begin{itemize}
				\item Seguridad a nivel de sistema: Las medidas de seguridad de sistema representan la última línea de defensa contra un problema de seguridad relacionado con Internet. Por lo tanto, el primer paso de una estrategia de seguridad en Internet completa debe ser configurar debidamente la seguridad básica del sistema.
				
				\item Seguridad a nivel de red: Las medidas de seguridad de red controlan el acceso sistemas de red. Cuando conecta la red a Internet, debe asegurarse de que tiene implantadas las debidas medidas de seguridad adecuadas a nivel de la red para proteger los recursos internos de la misma contra la intrusión y el acceso no autorizado. El medio más común para garantizar la seguridad de la red es un cortafuegos. El proveedor de servicios de Internet (ISP) puede proporcionar una parte importante del plan de seguridad de la red. El esquema de seguridad de la red debe indicar qué medidas de seguridad proporciona el ISP, como las reglas de filtrado de la conexión del direccionador del ISP y las medidas de precaución del sistema de nombres de dominio (DNS) público.
				
				\item Seguridad a nivel de aplicaciones: Las medidas de seguridad a nivel de aplicaciones controlan cómo pueden interactuar los usuarios con las aplicaciones. En general, tendrá que configurar valores de seguridad para cada una de las aplicaciones que utilice. Sin embargo, conviene que preste una atención especial al configurar la seguridad de las aplicaciones y los servicios que utilizará de Internet o que proporcionará a Internet. Estas aplicaciones y servicios son vulnerables al mal uso por parte de los usuarios no autorizados que buscan una manera de acceder a los sistemas de la red. Las medidas de seguridad que decida utilizar deberán incluir los riesgos del lado del servidor y del lado del cliente.
				
				\item Seguridad a nivel de transmisión: Las medidas de seguridad a nivel de transmisión protegen las comunicaciones de datos dentro de la red y entre varias redes. Cuando se comunica en una red que no es de confianza como Internet, no puede controlar cómo fluye el tráfico desde el origen hasta el destino. El tráfico y los datos transportados fluyen a través de distintos sistemas que están fuera de su control. A menos que configure medidas de seguridad como, por ejemplo, la configuración de las aplicaciones para utilizar TLS (Transport Layer Security), los datos direccionados están disponibles para que cualquiera pueda verlos y utilizarlos. Las medidas de seguridad a nivel de transmisión protegen los datos mientras fluyen entre los límites de otros niveles de seguridad.
				
			\end{itemize}
		\subsection{zero trust}
			Refiere a una estrategia de seguridad de red basada en la filosofía de que ninguna persona o dispositivo dentro o fuera de la red de una organización debe tener acceso para conectarse a sistemas hasta que se considere explícitamente necesario. 
	\section{Busque noticias sobre algún incidente de seguridad de la información de público conocimiento. Identifique cuál fue el bien afectado, qué error permitió el problema y qué se podría haber realizado para evitar el mismo.}
	Recientemente ocurrio un leak bastante grande a una empresa desarrolladora de videojuegos llamada Gamefreak.
	Esta empresa es conocida por ser la desarrolladora de los videojuegos de Pokemón. Entre todo el material filtrado se encontraba: Arte conceptual, prototipos de videojuegos, herramientas de desarrollo, repositorios utilizados por la empresa; información sobre futuros juegos, etc. mediante esta filtración en especifico también se conoció sobre los planes de Nintendo (empresa con la que están muy relacionados) sobre el plan de crear una nueva consola de videojuegos que reemplace a la Nintendo Switch.
	Este ejemplo funciona para relacionarlo con el concepto de confidencialidad ya que se divulgo mucho material que no se planeaba fuera público y que podría afectar a la empresa en imagen o economicamente.
	Se teoriza que la filtración se dio por un acceso no permitido a un servidor de archivos del que hacía uso la empresa. El acceso al mismo se dio por haber descubierto un gitlab privado en el que se publicaron accidentalmente credenciales de acceso para el servidor.
	Una forma de haber evitado esto podría haber sido agregar más seguridad (o seguridad por capas, podría pensarse) para el servidor de archivos, como filtros de direcciones IP, hacer uso de VPN, entre otras. 
	También tener herramientas para detección de filtrado de credenciales en el gitlab.
	
	\section{Analice los riesgos introducidos en una organización por el uso de redes sociales y mensajería instantánea. Relacione con ingeniería social y ciberinteligencia. Dé al menos un ejemplo que se haya judicializado.}
	
	Los riesgos introducidos a causa de estas aplicaciones son muchos. Es común las malas practicas en organizaciones en relación a información que debería ser privada y el uso de estas. Por ejemplo, empleados enviando credenciales mediante WhatsApp. En caso de que se roben las cookies de una sesión de WhatsApp web, un atacante podría ver la información compartida.
	
	
	
	
	
\end{document}
	